\begin{resumen}

%El desarrollo de videojuegos, especialmente del género RPG, supone el uso en conjunto de una serie de tópicos que llevan al límite los recursos disponibles para la consolidación del mismo. El diseño de argumentos y la generación de eventos en torno a este suponen una tarea sumamente importante pues tendrán un gran efecto en la experiencia del jugador. Debido a esto la construcción de un guión (o guiones) es una tarea en la cual se hace necesario invertir una gran cantidad de esfuerzo. Bajo esta premisa, se propone un sistema capaz de automatizar estas tareas al generar eventos relacionados a una historia central utilizando al mismo tiempo modelos de personalidad que permitan distinguir las preferencias del jugador. La aplicación de este sistema permitiría reducir los recursos destinados al diseño de un argumento aligerando a la vez la carga tras el desarrollo de un videojuego.

En el desarrollo de videojuegos, más específicamente en el género RPG, la historia (contenido argumental) es uno de los pilares fundamentales para el disfrute y entretenimiento del jugador \cite{schell2019art}. El paradigma actual de desarrollo implica una participación humana directa en la formulación de eventos y sucesos que hacen de esta tarea una de las más complejas de concretar. Actualmente técnicas tales como \textit{Story Assembler}, CONAN, \textit{Story and Discourse} facilitan la creación de dichos marcos argumentales a partir de eventos descritos previamente por un guionista. Estás técnicas son súmamente útiles cuando se desea reflejar la coherencia de un personaje por medio de su intencionalidad \cite{breault2018let}, pero no disponen de métodos apropiados para adaptar las historias resultantes a las preferencias del jugador. En el presente trabajo proponemos un modelo centrado en el uso de redes neuronales y agentes de planeamiento para la generación de historias en donde el uso de modelos de personalidad permita definir un marco argumental relativamente adaptable. 



\end{resumen}
