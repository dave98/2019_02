\begin{resumen}

El desarrollo de videojuegos, especialmente del género RPG, supone el uso en conjunto de una serie de tópicos que llevan al límite los recursos disponibles para la consolidación del mismo. Particularmente el diseño del argumento y la generación de eventos en torno a este suponen una tarea sumamente importante pues tendrán un gran efecto en la experiencia del jugador. Debido a esto la construcción de un guión (o guiones) es una tarea en la cual se hace necesario invertir una gran cantidad de esfuerzo. Bajo esta premisa, se propone un sistema capaz de automatizar estas tareas al generar eventos relacionados a una historia central utilizando al mismo tiempo modelos de personalidad que permitan distinguir las preferencias del jugador. La aplicación de este sistema permitiría reducir los recursos destinados al diseño de un argumento aligerando a la vez la carga tras el desarrollo de un videojuego.


\end{resumen}
