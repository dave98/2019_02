\begin{abstract}

\textit{Interactive Storytelling} no es un área reciente.  En áreas de estudio derivadas tales como \textit{Procedural Narrative} y \textit{Narrative Generation} existen una serie de modelos que proponen la creación de historias a partir de un marco argumental(\textit{StoryAssembler}) además de modelos capaces de crear \textit{quests} a partir de una historia (\textit{CONAN}). Si bien los enfoques de ambos modelos difieren ligeramente, bajo la presente investigación se propone un sistema híbrido capaz de manejar ambas facilidades. El sistema propuesto cuenta además con un módulo de adaptación al jugador. Dicho módulo maneja una serie de variables en tiempo real que por medio de una red neuronal, permiten definir un modelo de comportamiento en el jugador (\textit{The Big Five Model}). Dicho modelo es posteriormente utilizado para adaptar la generación de \textit{quests} hacia las inclinaciones del jugador. 


\end{abstract}
