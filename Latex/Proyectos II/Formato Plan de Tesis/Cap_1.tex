\chapter{Introducción}

%Este es el primer capítulo de la tesis. Se inicia con el desarrollo de la introducción de la tesis. Es importante que el texto utilice la tabla de abreviaturas correctamente. En el archivo abreviaturas.tex contiene la tabla de abreviaturas. Para citar alguna de ellas debes usar los comandos $\backslash$ac\{tu-sigla-aqui\}. Si es la primera vez que utilizas la sigla ella se expandirá por completo. Por ejemplo, el comando $\backslash$ac\{CMM\} va a producir: \ac{CMM}. Si más adelante repites el mismo comando sólo aparecerá la sigla \ac{CMM}. Para explorar mucho más este comando es necesario leer su manual disponible en: $http://www.ctan.org/tex-archive/macros/latex/contrib/acronym/$


\section{Motivación y Contexto}

%En esta sección se va desde aspectos generales a  aspectos específicos (como un embudo). No se olvide que es la primera parte que tiene contacto con el lector y que hará que este se interese en el tema a investigar. El objetivo de esta sección es llevar al lector hacie el tema que se va a tratar en forma específica y dejar la puerta abierta a otras investigaciones

Con la acelerada expansión del mundo de los videojuegos y su cada vez mayor influencia en la vida de sus jugadores, se hace necesario comprender que factores son súmamente influyentes en el éxito de una franquicia. Dentro del género \ac{RPG}, videojuegos tales como: \textit{The Elder Scrolls: Skyrim}, \textit{Fallout 3}, \textit{The Witcher}, \textit{Dying Light}, etc han llegado a alcanzar estatus de culto debido al impecable manejo de cuatro pilares fundamentales: Mecánicas, Estética, Tecnología e Historia. 

Si bien no existe un concepto más importante que otro (al menos durante la etapa de desarrollo), destacaremos el ámbito de la historia al ser uno de los más influyentes en la experiencia del jugador y por ser además el enfoque principal de esta investigación. Contar historias no es una tarea fácil. Al crear y narrar una histora se hace uso de la Inteligencia Narrativa. El propósito de esta es trasmitir una experiencia formada a partir de sucesos reales o ficticios. Dentro del mundo de los videojuegos, las historias representan el eje central en torno al cual giran las experiencias propias del jugador (especialmente en videojuegos del género \ac{RPG}). Una historia puede a la vez ser dividida en pequeños fragmentos que separados conforman las \textit{quest}(misiones) que el jugador tendrá superar para llegar a la completitud de la historia principal. Estos \textit{quest} son también el punto principal de partida para argumentos secundarios que pueden llegar a alargar un poco más la vida útil de un videojuego. 

A pesar de su importancia, la aplicación de \textit{quests} en un videojuego conlleva a la realización de una tarea titánica. Centrados ya en el género \ac{RPG}, es casi  común que la mayoría de \textit{quests} a resolverse (ya sean principales o secundarios) hayan sido desarrollado bajo la supervisión de un guionista. Esto por lo general, es lo más adecuado. Sin embargo, el problema se origina cuando la cantidad de \textit{quest} a querer desarrollarse sobrepasa por mucho la capacidad de los guionistas disponibles. Lo natural, llegados a este punto, sería plantear una pronta delimitación en la cantidad de \textit{quest} a desarrollarse, pero esto también repercutiría en la vida útil de nuestro producto. Se sabe que juegos como \textit{The Elder Scrolls: Skyrim} posee alrededor de 244 misiones y que son contados los jugadores que lograron completarlas en su totalidad. El punto entonces no es llenar un videojuego de \textit{quest} pero tampoco presentarlos en una cantidad muy escasa.

\section{Planteamiento del Problema}

Crear historias y diseñar \textit{quests} en videojuegos del género RPG es una tarea muy laboriosa. Reducir el margen de autoría mediante la utilización sistemas como Cardinal facilitaría en gran medida el proceso de desarrollo de un videojuego.


\section{Objetivos}

%1En esta sección se colocan los objetivos generales de la tesis. Máximo dos. Si necesita ampliar estos objetivos utilice la sección de objetivos específicos.
Cardinal es una propuesta centrada en aligerar la carga tras los diseños argumentales de un videojuego. Esto no implica prescindir de un guionista o escritor, sino más bien reducir el esfuerzo autorial que este realizaría al desarrollar enteramente un argumento.

\subsection{Objetivos Específicos}

%En esta sección se coloca el los objetivos específicos de la tesis, que serán aquellos que contesten a lasinterrogantes de investigación.
Mediante el uso de HTNs proponemos un sistema capaz de interpretar un marco argumental, y que partir de este desarrolle y genere los quest que conforman un videojuego. 

Cardinal también es una propuesta centrada en aprender del jugador. Apoyándose en el desempeño del mismo, genera  \textit{quests} tomando como parámetro modelos de comportamiento y personalidad reconocidos mediante el uso de redes neuronales profundas.



\section{Organización del Plan de  Tesis}

En esta sección se coloca cuantos capítulos contendrá la tesis y que se tratará en cada uno de
ellos en forma resumida. Dedíquele un párrafo de dos o tres líneas a explicar cada capítulo.

