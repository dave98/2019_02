\chapter{Introducción}

%Este es el primer capítulo de la tesis. Se inicia con el desarrollo de la introducción de la tesis. Es importante que el texto utilice la tabla de abreviaturas correctamente. En el archivo abreviaturas.tex contiene la tabla de abreviaturas. Para citar alguna de ellas debes usar los comandos $\backslash$ac\{tu-sigla-aqui\}. Si es la primera vez que utilizas la sigla ella se expandirá por completo. Por ejemplo, el comando $\backslash$ac\{CMM\} va a producir: \ac{CMM}. Si más adelante repites el mismo comando sólo aparecerá la sigla \ac{CMM}. Para explorar mucho más este comando es necesario leer su manual disponible en: $http://www.ctan.org/tex-archive/macros/latex/contrib/acronym/$


\section{Motivación y Contexto}

%En esta sección se va desde aspectos generales a  aspectos específicos (como un embudo). No se olvide que es la primera parte que tiene contacto con el lector y que hará que este se interese en el tema a investigar. El objetivo de esta sección es llevar al lector hacie el tema que se va a tratar en forma específica y dejar la puerta abierta a otras investigaciones

Con la acelerada expansión del mundo de los videojuegos y su cada vez mayor influencia en la vida de sus jugadores, se hace necesario comprender que factores son súmamente influyentes en el éxito de una franquicia. Dentro del género \ac{RPG}, videojuegos tales como: \textit{The Elder Scrolls: Skyrim}, \textit{Fallout 3}, \textit{The Witcher}, \textit{Dying Light}, etc han llegado a alcanzar estatus de culto debido al impecable manejo de cuatro pilares fundamentales: Mecánicas, Estética, Tecnología e Historia. 

Si bien no existe un concepto más importante que otro (al menos durante la etapa de desarrollo), destacaremos el ámbito de la historia al ser uno de los más influyentes en la experiencia del jugador y por ser además el enfoque principal de esta investigación. Contar historias no es una tarea fácil. Al crear y narrar una histora se hace uso de la Inteligencia Narrativa. El propósito de esta es trasmitir una experiencia formada a partir de sucesos reales o ficticios. Dentro del mundo de los videojuegos, las historias representan el eje central en torno al cual giran las experiencias propias del jugador (especialmente en videojuegos del género \ac{RPG}).


\section{Planteamiento del Problema}

En esta sección se realiza el planteamiento del problema que queremos resolver con la tesis. Sea muy puntual y no ocupe más de un párrafo en especificar cual es el problema que desea atacar.

\section{Objetivos}

En esta sección se colocan los objetivos generales de la tesis. Máximo dos. Si necesita ampliar estos objetivos utilice la sección de objetivos específicos.

\subsection{Objetivos Específicos}

En esta sección se coloca el los objetivos específicos de la tesis, que serán aquellos que contesten a las
interrogantes de investigación.

\section{Organización del Plan de  Tesis}

En esta sección se coloca cuantos capítulos contendrá la tesis y que se tratará en cada uno de
ellos en forma resumida. Dedíquele un párrafo de dos o tres líneas a explicar cada capítulo.

\section{Cronograma}

Esta sección sólo es para aquellos alumnos que estén presentando su plan de tesis. Esta sección no va
en la tesis final.
