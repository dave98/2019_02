\chapter{CARDINAL}

%En este capítulo se desarrolla toda la propuesta realizada a través de la investigación. Sigue la misma estructura del capítulo anterior.
%El título del capítulo es flexible de acuerdo a cada tesis. Algunos títulos sugeridos podrían ser:
%Este título debe de estar ade acuerdo con el asesor del tema. Consúltelo en su sala de clase.

CARDINAL es una propuesta enfocada en la utilización de cuatro pilares claves. 

El primer pilar implica la identificación de un modelo de personalidad-comportamiento que evolucione conforme el jugador va desenvolviéndose a lo largo del juego. Dicho modelo se obtiene mediante la aplicación de dos técnicas:
\begin{itemize}
\item \textit{The Big Five Test} es un cuestionario de 10 preguntas que permite identificar un modelo básico de personalidad al establecer una serie de escenarios que al ser resueltos, definen los parámetros iniciales de \textit{Openess}, \textit{Extraversion}, \textit{Neuroticism}, \textit{Conscientiousness} y \textit{Agreeableness}. REF14

\item La segunda técnica es una propuesta hecha por REF15. Mediante el uso de una red neuronal entrenada bajo variables relacionadas a las mecánicas del juego, se hace posible la obtención de un modelo de comportamiento. De manera similar al modelo personalidad, el modelo de comportamiento también se define bajo los estándares de \textit{The Big Five} con la diferencia de que este modelo es dinámico. - Extension de descripción -
\end{itemize}

Finalmente, la obtención del modelo personalidad-comportamiento se obtiene de la ponderación de los modelos previos. Esta última etapa permite la aplicación de una variable de influencia que permita asignar el grado de perticipación que tiene cada modelo componente. En este caso, y debido a la naturaleza del videojuego en el que nos efocamos (RPG), se dará prioridad al modelo de personalidad. 

El segundo pilar implica el uso de un planeador \ac{DPOCL}





